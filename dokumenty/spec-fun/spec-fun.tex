\documentclass[a4paper,12pt]{article}
\newcommand\tab[1][0.6cm]{\hspace*{#1} }
\usepackage[T1]{fontenc}
\usepackage[polish]{babel}
\usepackage[utf8]{inputenc}
\usepackage{lmodern}
\usepackage{hyperref}
\usepackage[top=2cm, bottom=2cm, left=2cm, right=2cm]{geometry}
\usepackage{listings}
\usepackage{amsmath}
\usepackage{graphicx}
\usepackage{float}
\usepackage{fancyhdr}
\usepackage{lastpage}

\title{ \sc{Specyfikacja funkcjonalna} \\
\emph{Projekt zespołowy} }

\author{Łukasz Knigawka \and Mateusz Smoliński}

\begin{document}

\maketitle

\thispagestyle{empty}

\tableofcontents

\newpage

\section{Wstęp teoretyczny}

\subsection{Słownik pojęć}

\tab Zapoznanie się z poniższymi pojęciami ułatwi zrozumienie dokumentu. 
\\\textit{Obszar} -- fragment terenu, na którym znajduje się tylko jeden punkt kluczowy
\\\textit{Punkt kluczowy} -- punkt, według którego współrzędnych wyznaczane są obszary
\\\textit{Obiekt} -- element terenu podlegający analizie
\\\textit{Budynek mieszkalny} -- obiekt zawierający informację o liczbie jego mieszkańców

\subsection{Założenia}

\begin{enumerate}
\item Pod danymi współrzędnymi może znajdować się tylko jeden punkt kluczowy lub obiekt.
\item Każdy punkt kluczowy lub obiekt zajmuje dokładnie jedną jednostkę terenu.
\item Każdy punkt kluczowy lub obiekt musi znajdować się na co najmniej jednym obszarze. Znajduje się on na więcej niż jednym obszarze, gdy leży na granicy obszarów.
\end{enumerate}

\section{Funkcje programu}

\tab Program posiada trzy główne funkcje:
\begin{enumerate}
\item rysowanie planszy na podstawie danych z pliku,
\item modyfikowanie planszy za pomocą prostych narzędzi z przybornika,
\item pokazywanie list mieszkańców oraz obiektów na wybranym przez użytkownika obszarze.
\end{enumerate}
\tab Użytkownik otrzymuje powyższe główne funkcje dzięki poniższym funkcjom pomocniczym:
\begin{itemize}
\item wczytanie danych z pliku tekstowego,
\item wczytanie grafiki tła,
\item dodawanie elementów konturu planszy,
\item dodawanie punktów kluczowych dzielących kontur na obszary,
\item dodawanie obiektów o typach zadeklarowanych w pliku tekstowym,
\item wypisywanie list z liczbą mieszkańców na danym obszarze,
\item wypisywanie list obiektów dla danego obszaru,
\item możliwość pogrupowania obiektów z tej listy,
\item wyświetlenie pomocy informującej użytkownika o wszystkich funkcjach programu oraz o tym, jak należy ich używać,
\item obsługa możliwych błędów oraz informowanie o ich rodzajach.
\end{itemize}

\section{Opis interakcji użytkownika z programem}

\tab Na \textit{Rysunku 1} przedstawiono prototypowy wygląd okna głównego programu. W górnej części okna znajduje się pasek narzędzi, na którym umieszczono trzy szare przyciski. Kliknięcie pierwszego z nich otwiera eksplorator plików umożliwiający wybranie pliku tekstowego, z którego zostaną wczytane dane. Wybranie drugiego przycisku skutkuje otwarciem eksploratora umożliwiającego wybór pliku graficznego, który stanie się tłem dla planszy. Program oczekuje na plik graficzny w foramcie \textit{.png}. Kliknięcie trzeciego przycisków wywoła wyświetlenie w polach tekstowych, znajdujących się po prawej stronie mapy, instrukcji obsługi programu. Na pasku narzędzi umieszczonym na górze okna znajdują się także dwa kolorowe przyciski-przełączniki. Tylko jeden z nich może być wybrany w danej chwili, domyślnie jest to przycisk-przełącznik oznaczony etykietą \textit{Kontur}.
W zależności od tego który z przycisków został wybrany, taki rodzaj elementu będzie dodawany na mapę po kliknięciu na niej lewym przyciskiem myszy. Element będzie dodany w miejscu w którym znajdował się kursor w momencie kliknięcia. Kliknięcie prawym przyciskiem myszy na punkt konturu lub punkt kluczowy usuwa dany punkt.
\\\tab Poniżej, w centrum okna głównego, znajduje się mapa terenu, na której wyświetlane mogą być kontury terenu, punkty kluczowe oraz obiekty.
\\\tab Po prawej stronie mapy znajdują się dwa pola tekstowe. Pierwsze z nich, wraz z najechaniem na dowolny punkt kluczowy, ukazuje listę obiektów danego obszaru. Pole umiejscowione niżej, po najechaniu przez użytkownika kursorem myszy na dowolny punkt kluczowy, wyświetla listę obiektów znajdujących się w danym obszarze, z grupowaniem po typie. Te dwa pola tekstowe są także wykorzystywane przy wyświetlaniu pomocy.
\\\tab W dolnej części okna głównego znajduje się pasek stanu, na którym wyświetlane są komunikaty o błędach.

\begin{figure}[H]
\centering
\includegraphics[width=18cm] a
\caption{Okno główne programu}
\end{figure}

Poniżej przedstawiono przykładowy plik wejściowy. Program oczekuje na plik tekstowy w formacie \textit{.txt}.

\begin{lstlisting}
# Kontury terenu (wymienione w kolejnosci laczenia): Lp. x y
1. 0 0
2. 0 20
3. 20 30.5
4. 40 20
5. 40 0

# Punkty kluczowe: Lp. x y Nazwa
1. 1 1 KOK Krawczyka
2. 1 19 KOK Kaczmarskiego
3. 30 10 KOK Lazarewicza

# Definicje obiektow: Lp. Typ_obiektu (Nazwa_zmiennej Typ_zmiennej)*
1. SZKOlA Nazwa String X double Y double
2. DOM X double Y double L_MIESZKANCOW int
3. NIEDZWIEDZ X double Y double

# Obiekty: Typ_obiektu (zgodnie z definicja)
1. SZKOlA "Szkola robienia duzych pieniedzy" 4 1
2. DOM 4 3 100
3. DOM 4 17 120
4. DOM 4 18 80
5. NIEDZWIEDZ 20 20
6. NIEDZWIEDZ 40 1
7. NIEDZWIEDZ 39 1
8. NIEDZWIEDZ 39 2
\end{lstlisting}

Strukturę pliku wejściowego można podzielić na 4 sekcje, gdzie każda zaczyna się od niezmiennej linii komentarza, rozpoczynającej się od znaku \textit{\#}.

\section{Sytuacje wyjątkowe}

\tab Jednym z najważniejszych aspektów programu jest obsługa sytuacji wyjątkowych i błędów. Mogą być one związane z niepoprawnym podaniem danych przez użytkownika, podaniem danych nieobsługiwanych przez program lub próbą podjęcia niewłaściwych czynności. Poniżej znajduje się lista sytuacji wyjątkowych przewidzianych w tym projekcie, wraz ze sposobem, w jaki program reaguje na te zdarzenia.

\begin{enumerate}

\item Sytuacje związane z tekstowym plikiem wejściowym:
\begin{itemize}
\item gdy użytkownik wskaże na nieistniejący plik, program wyświetli okno dialogowe z informacją o nieodnalezieniu żądanego pliku,
\item gdy plik będzie pusty lub nie będzie złożony z czterech części oddzielonych liniami rozpoczynającymi się znakiem \#, program wyświetli komunikat o niewłaściwej strukturze pliku wejściowego,
\item gdy plik wejściowy będzie zawierać linię przechowującą zbyt mało argumentów, lub gdy jeden z tych argumentów będzie niepoprawnie podany (np. argument, który miał być liczbą będzie zawierać literę, lub zawiera ujemną wartość), program wyświetli komunikat o błędzie odczytywania linii wraz z podaniem numeru danej linii,
\item gdy jeden z parametrów x, y będzie wykraczać poza ustalony maksymalny rozmiar mapy (600x600 pikseli), program wyświetli okno dialogowe z informacją przekroczeniu limitu rozmiaru w danej linii,
\item gdy w deklaracji konturów wystąpią przynajmniej dwie przecinające się linie, program wyświetli komunikat o przecinaniu się linii konturowych,
\item gdy przy deklaracji punktów kluczowych przynajmniej jeden z nich znajduje się poza granicami wyznaczonymi wcześniej jako kontury planszy, program wyświetli okno dialogowe z informacją o niewłaściwym punkcie kluczowym wraz z podaniem numeru danej linii,
\item gdy w deklaracji obiektów nie będą znajdować się trzy obowiązkowe typy obiektów -- dom, szkoła oraz niedźwiedź, program wyświetli komunikat o braku obowiązkowych deklaracji obiektów,
\item gdy w linii zawierającej deklarację obiektu nie zostanie rozpoznany typ parametru obiektu, zostanie wyświetlone okno dialogowe o nierozpoznaniu typu parametru wraz z podaniem linii, w której wystąpił problem,
\item gdy w ostatniej części pliku, zawierającej listę obiektów znajdzie się obiekt o niezadeklarowanym wcześniej typie, program wyświetli komunikat o nieznanym typie obiektu w danej linii,
\item gdy na liście obiektów zostanie umieszczony obiekt znajdujący się poza granicami wyznaczonymi przez kontur mapy, zostanie wyświetlony komunikat o niewłaściwym obiekcie w danej linii.

\end{itemize}

\item Sytuacje związane z interakcją z programem
\begin{itemize}
\item gdy nie uda się otworzyć pliku graficznego, wyświetlone zostanie okno dialogowe z informacją o nieudanej próbie otwarcia pliku graficznego,
\item gdy użytkownik próbuje dodać punkt kluczowy poza konturem planszy, program wyświetli komunikat o niemożliwości dodania punktu w tym miejscu.
\end{itemize}
\end{enumerate}

\section{Zarys testów akceptacyjnych}

\tab Test akceptacyjne zostaną przeprowadzone bez użycia zewnętrznych narzędzi. Ich celem będzie uruchomienie i sprawdzenie poprawności działania programu dla różnych danych wejściowych oraz różnych sposobów interakcji z programem.

Pierwsze testy akceptacyjne będą polegać na uruchomieniu programu z różnymi, poprawnie przygotowanymi zestawami danych. W szczególności przeprowadzone zostaną testy z zastosowaniem plików:

\begin{itemize}
\item z niewielką ilością punktów granicznych (mniej niż 5 elementów konturu),
\item z dużą ilością punktów granicznych (więcej niż 20 elementów konturu),
\item bez deklaracji dodatkowych typów obiektów,
\item z deklaracjami dodatkowych typów obiektów,
\item o minimalnym i maksymalnym rozmiarze planszy,
\item graficznych o minimalnym i maksymalnym rozmiarze planszy.
\end{itemize}

Przeprowadzone zostaną także testy mające na celu sprawdzenie odporności programu na błędne dane wejściowe. W tej części program zostanie sprawdzony pod kątem plików:

\begin{itemize}
\item z błędnie zapisanymi danymi, omówionymi w sytuacjach wyjątkowych,
\item zawierających kontury z przecinającymi się liniami,
\item o rozmiarze planszy przekraczającym górny limit
\item niezawierających żadnych punktów,
\item próbujących umieścić punkty kluczowe lub obiekty poza granicami planszy,
\item graficznych o zbyt dużym rozmiarze.
\end{itemize}

W trakcie trwania programu sprawdzone zostaną także próby wpisania niepoprawnych danych przy dodawaniu obiektów oraz prób ręcznego dodania punktów kluczowych poza granicami planszy.

\end{document}
