\documentclass[a4paper,12pt]{article}
\newcommand\tab[1][0.6cm]{\hspace*{#1} }
\usepackage[T1]{fontenc}
\usepackage[polish]{babel}
\usepackage[utf8]{inputenc}
\usepackage{lmodern}
\usepackage{hyperref}
\usepackage[top=2cm, bottom=2cm, left=2cm, right=2cm]{geometry}
\usepackage{listings}
\usepackage{amsmath}
\usepackage{graphicx}
\usepackage{float}
\usepackage{fancyhdr}
\usepackage{lastpage}

\title{ \sc{Specyfikacja funkcjonalna} \\
\emph{Projekt zespołowy} }

\author{Łukasz Knigawka \and Mateusz Smoliński}

\begin{document}

\maketitle

\thispagestyle{empty}

\tableofcontents

\newpage

\section{Słownik pojęć}

\tab Zrozumienie dokumentu ułatwia zapoznanie się z poniższymi definicjami.
\\\textit{Punkt kluczowy} -- bank, parabank lub kasa oszczędnościowo-kredytowa (\textit{KOK})
\\\textit{Obszar} -- fragment terenu, na którym znajduje się tylko jeden punkt kluczowy

\section{Funkcje programu}

\tab Program umożliwia użytkownikowi:
\begin{enumerate}
\item wczytanie z pliku tekstowego (\textit{.txt}) konturu terenu, współrzędnych punktów kluczowych, informacji o obiektach,
\item graficzne przedstawienie konturu terenu, punktów kluczowych, optymalnych granic obszarów,

\end{enumerate}

\section{Opis interakcji użytkownika z programem}



\section{Sytuacje wyjątkowe}

\section{Zarys testów akceptacyjnych}

\end{document}
