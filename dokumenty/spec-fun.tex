\documentclass[a4paper,12pt]{article}
\newcommand\tab[1][0.6cm]{\hspace*{#1} }
\usepackage[T1]{fontenc}
\usepackage[polish]{babel}
\usepackage[utf8]{inputenc}
\usepackage{lmodern}
\usepackage{hyperref}
\usepackage[top=2cm, bottom=2cm, left=2cm, right=2cm]{geometry}
\usepackage{listings}
\usepackage{amsmath}
\usepackage{graphicx}
\usepackage{float}
\usepackage{fancyhdr}
\usepackage{lastpage}

\title{ \sc{Specyfikacja funkcjonalna} \\
\emph{Projekt zespołowy} }

\author{Łukasz Knigawka \and Mateusz Smoliński}

\begin{document}

\maketitle

\thispagestyle{empty}

\tableofcontents

\newpage

\section{Wstęp teoretyczny}

\subsection{Słownik pojęć}

\tab Zrozumienie dokumentu ułatwia zapoznanie się z poniższymi definicjami.
\\\textit{Punkt kluczowy} -- bank, parabank lub kasa oszczędnościowo-kredytowa (\textit{KOK})
\\\textit{Obszar} -- fragment terenu, na którym znajduje się tylko jeden punkt kluczowy
\\\textit{Obiekt} -- Element terenu podlegający analizie
\\\textit{Budynek mieszkalny} -- Obiekt zawierający informację o liczbie jego mieszkańców

\subsection{Założenia}

\begin{enumerate}
\item Pod danymi współrzędnymi może znajdować się tylko jeden punkt kluczowy lub obiekt. 
\item Każdy punkt kluczowy lub obiekt zajmuje dokładnie jedną jednostkę terenu. 
\item Każdy punkt kluczowy lub obiekt musi znajdować się na co najmniej jednym obszarze. Znajduje się on na więcej niż jednym obszarze, gdy leży na granicy obszarów.
\end{enumerate}

\section{Funkcje programu}

\tab Program, na podstawie danych wejściowych, umożliwia użytkownikowi:
\begin{enumerate}
\item wczytanie z pliku tekstowego (\textit{.txt}) konturu terenu, współrzędnych punktów kluczowych, informacji o obiektach,
\item graficzne przedstawienie konturu terenu, punktów kluczowych, optymalnych granic obszarów.
\end{enumerate}

\tab W trakcie działania programu, program umożliwia użytkownikowi:
\begin{enumerate}
\item dodawanie lub usuwanie konturu terenu,
\item dodawanie lub usuwanie punktów kluczowych, 
\item nakładanie grafiki pod wyznaczone kontury,
\item wyświetlenie listy obiektów należących do danego obszaru z grupowaniem po typie,
\item wyświetlenie liczby mieszkańców wybranego obszaru, na podstawie informacji o budynkach mieszkalnych na danym obszarze.
\end{enumerate}

\section{Opis interakcji użytkownika z programem}



\section{Sytuacje wyjątkowe}

\section{Zarys testów akceptacyjnych}

\end{document}
