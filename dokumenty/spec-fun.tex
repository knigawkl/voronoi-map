\documentclass[a4paper,12pt]{article}
\newcommand\tab[1][0.6cm]{\hspace*{#1} }
\usepackage[T1]{fontenc}
\usepackage[polish]{babel}
\usepackage[utf8]{inputenc}
\usepackage{lmodern}
\usepackage{hyperref}
\usepackage[top=2cm, bottom=2cm, left=2cm, right=2cm]{geometry}
\usepackage{listings}
\usepackage{amsmath}
\usepackage{graphicx}
\usepackage{float}
\usepackage{fancyhdr}
\usepackage{lastpage}

\title{ \sc{Specyfikacja funkcjonalna} \\
\emph{Projekt zespołowy} }

\author{Łukasz Knigawka \and Mateusz Smoliński}

\begin{document}

\maketitle

\thispagestyle{empty}

\tableofcontents

\newpage

\section{Wstęp teoretyczny}

todo


\section{Funkcje programu}

\tab Program posiada trzy główne funkcje:
\begin{enumerate}
\item rysowanie planszy na podstawie danych z pliku,
\item modyfikowanie planszy za pomocą prostych narzędzi z przybornika,
\item pokazywanie list mieszkańców oraz obiektów na wybranym przez użytkownika obszarze.
\end{enumerate}
\tab Użytkownik otrzymuje powyższe główne funkcje dzięki poniższym funkcjom pomocniczym:
\begin{itemize}
\item wczytanie danych z pliku tekstowego,
\item wczytanie grafiki tła,
\item dodawanie elementów konturu planszy,
\item dodawanie punktów kluczowych dzielących kontur na obszary,
\item dodawanie obiektów o typach zadeklarowanych w pliku tekstowym,
\item wypisywanie list z liczbą mieszkańców na danym obszarze,
\item wypisywanie list obiektów dla danego obszaru,
\item możliwość pogrupowania obiektów z tej listy,
\item wyświetlenie pomocy informującej użytkownika o wszystkich funkcjach programu oraz o tym, jak należy ich używać,
\item obsługa możliwych błędów oraz informowanie o ich rodzajach.
\end{itemize}

\section{Opis interakcji użytkownika z programem}

Każdy wie jak obsługiwać mikrofalówkę

\section{Sytuacje wyjątkowe}

Wszystkie są na swój sposób wyjątkowe

\section{Zarys testów akceptacyjnych}

\tab Test akceptacyjne zostaną przeprowadzone bez użycia zewnętrznych narzędzi. Ich celem będzie uruchomienie i sprawdzenie poprawności działania programu dla różnych danych wejściowych oraz różnych sposobów interakcji z programem.

Pierwsze testy akceptacyjne będą polegać na uruchomieniu programu z różnymi, poprawnie przygotowanymi zestawami danych. W szczególności przeprowadzone zostaną testy z zastosowaniem plików:

\begin{itemize}
\item z niewielką ilością punktów granicznych (mniej niż 5 elementów konturu),
\item z dużą ilością punktów granicznych (więcej niż 20 elementów konturu),
\item bez deklaracji dodatkowych typów obiektów,
\item z deklaracjami dodatkowych typów obiektów,
\item o minimalnym i maksymalnym rozmiarze planszy,
\item graficznych o minimalnym i maksymalnym rozmiarze planszy.
\end{itemize}

Przeprowadzone zostaną także testy mające na celu sprawdzenie odporności programu na błędne dane wejściowe. W tej części program zostanie sprawdzony pod kątem plików:

\begin{itemize}
\item z błędnie zapisanymi danymi, omówionymi w sytuacjach wyjątkowych,
\item zawierających kontury z przecinającymi się liniami,
\item o rozmiarze planszy przekraczającym górny limit
\item niezawierających żadnych punktów,
\item próbujących umieścić punkty kluczowe lub obiekty poza granicami planszy,
\item graficznych o zbyt dużym rozmiarze.
\end{itemize}

W trakcie trwania programu sprawdzone zostaną także próby wpisania niepoprawnych danych przy dodawaniu obiektów oraz prób ręcznego dodania obiektów i punktów kluczowych poza granicami planszy.

\end{document}
