\documentclass[a4paper,11pt]{article}
\newcommand\tab[1][0.6cm]{\hspace*{#1}}
\usepackage[T1]{fontenc}
\usepackage[polish]{babel}
\usepackage[utf8]{inputenc}
\usepackage{lmodern}
\usepackage{hyperref}
\usepackage[top=2cm, bottom=2cm, left=2cm, right=2cm]{geometry}
\usepackage{listings}
\usepackage{amsmath}
\usepackage{graphicx}
\usepackage{float}
\usepackage{fancyhdr}
\usepackage{lastpage}

\rfoot{\thepage \hspace{lpt} / \pageref{LastPage}}

\title{ \sc{Specyfikacja implementacyjna} \\
\emph{Projekt zespołowy} }

\author{Mateusz Smoliński \and Łukasz Knigawka}

\begin{document}

\maketitle

\thispagestyle{empty}

\tableofcontents

\newpage

\section{Wstęp}

\tab Celem projektu jest napisanie programu prezentującego interaktywną mapę na podstawie danych z pliku wejściowego. Program otrzymuje punkty tworzące kontur mapy, punkty kluczowe, na podstawie których kontur zostaje podzielony na obszary, a także definicje obiektów oraz listę takowych obiektów, które mają być brane pod uwagę przy analizie wyżej wspomnianych obszarów.

Jeśli dane będą podane prawidłowo, program na ich podstawie generuje planszę, która dalej może być modyfikowana przez użytkownika. Może on dodawać i usuwać elementy konturu, zmieniając kształt mapy, a także dodawać i usuwać punkty kluczowe, wpływając tym samym na podział na obszary. Po kliknięciu na jeden z punktów kluczowych użytkownik otrzyma informacje o obiektach znajdujących się na wyznaczonym przez niego obszarze.

Projekt ,,LUPA'' zostanie napisany w języku C\# wersji 7.3 (.NET Framework 4.7.2), w środowisku Microsoft Visual Studio 15.9.4. Interfejs użytkownika zostanie utworzony dzięki platformie WPF. Implementacja oraz testowanie programu odbywać się będą na komputerach o następujących parametrach: 
\begin{itemize}

\item 64-bitowy system operacyjny Windows 10 Home ver. 1803,
\item procesor Intel Core i5-7200U,
\item pamięć RAM 8,00 GB,
\item karty graficzne Intel HD Graphics 620 + NVIDIA GeForce 940MX,

\end{itemize}

oraz:

\begin{itemize}

\item 64-bitowy system operacyjny Windows 10 Home ver. 1803,
\item procesor Intel Core i7-6700HQ,
\item pamięć RAM 16,00 GB,
\item karty graficzne Intel HD Graphics 530 + NVIDIA GeForce 940MX,

\end{itemize}

\section{Diagram klas}

\tab Program składać się będzie z 9 klas. Zależności pomiędzy poszczególnymi klasami obrazuje Rysunek 1.

Na diagramie klas nie zostały uwzględnione:

\begin{itemize}
\item zależności pomiędzy klasami Parser, MainWindow, AreaDivider oraz klasami będącymi składowymi klasy Map, w celu poprawy czytelności,
\item metody wygenerowane przez WPF, odpowiadające za rysowanie na planszy oraz reakcję na zachowanie użytkownika.
\end{itemize}

Tu dopisać o tych właściwościach.

\begin{figure}[H]
\centering
\includegraphics[width=15cm]c
\caption{Diagram klas}
\label{fig:obrazek c}
\end{figure}

\newpage

\section{Opis klas/metod}


\subsection{Klasa MainWindow}

\tab Jest to główna klasa tego programu. opisać ją

\subsubsection{Metoda 1}


\tab Coś to robi

\subsubsection{Metoda 2}


\tab Pq

\subsection{Klasa 2}

\tab wq

\subsubsection{Metoda 1}


\tab Mxq


\subsection{Klasa 3}
\tab 

\subsubsection{metoda}
\begin{itemize}
\item \begin{lstlisting}
public CurrencyMatrix (int n)
\end{lstlisting}
\end{itemize}

\tab standard listingu

\subsection{Klasa 4}

\tab coś robi


\subsection{Klasa 5}

\tab xxx

\subsection{Klasa 6}

\tab W tej klasie coś

\subsubsection{Metoda 1}



\subsubsection{Metoda 2}



\section{Działanie programu i zastosowanie algorytmu}

\tab Pełna sekwencja. Opis algorytmu może być problematyczny na tym etapie

\section{Testy jednostkowe}

\tab Projekty testów zakładają użycie narzędzia ???. Poniżej znajduje się lista testów planowanych dla kluczowych metod programu, dla każdej z nich przewidziane są różne przypadki otrzymanych danych. 

\subsection{Testy klas 1 i 2}

\tab W tych dwóch nie trzeba wiele testować

\subsection{Testy klasy 3}

\tab coś trzeba tu stestować

\subsection{Testy klasy 4}

\tab Tu testujemy coś


\subsection{Testy klasy 5}

\tab Ta klasa ma coś

\subsection{Testy klasy 6}

\tab Ta klasa odpowiada za coś



\end{document}